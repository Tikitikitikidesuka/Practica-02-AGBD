%----------------------------------------------------------------------------------------
%	PAQUETES Y CONFIGURACIÓN
%----------------------------------------------------------------------------------------

\documentclass[a4paper, 11pt, oneside]{article} % Hoja A4, fuente 11pt y oneside

\newcommand{\scriptdir}{../scripts/} % Directorio de los scripts
\newcommand{\imagedir}{../images/} % Directorio de las imagenes
\newcommand{\plogo}{\fbox{$\mathcal{LCDPM}$}} % Logo del publisher

\usepackage[parfill]{parskip} % Para no indentar la primera linea de cada párrafo
\usepackage[utf8]{inputenc} % Para permitir el input de carácteres internacionales
\usepackage[T1]{fontenc} % Para permitir el output de carácteres internacionales
\usepackage{hyperref} % Para poner links en el índice
\usepackage{graphicx} % Para incluir imágenes
\usepackage{xcolor} % Para poner colores 
\usepackage{minted} % Para poner código
\usepackage{float} % Para fijar las tablas

\tolerance=9999			% Permite espacios en blanco grandes entre palabras
\emergencystretch=10pt		% Permite un poco de flexibilidad en la longitud de linea
\hyphenpenalty=10000		% Deshabilita los guiones para cortar palabras
\exhyphenpenalty=100		% Permite usar guiones puestos por el usuario

\definecolor{MintedBG}{HTML}{F2EEE9} % Color de fondo de las cajas minted

\setlength{\fboxsep}{0pt} % Quitar margen de color de las cajas minted

\setminted[]{
	frame=single,		% Caja alrededor del código
	framesep=8pt,		% Margen entre la caja y el código
	bgcolor=MintedBG,	% Color de fondo
	breaklines,		% Permitir cortar la linea en un linebreak
	breakafter=_,		% Permitir cortar palabras en las barras bajas
}

\setmintedinline[]{
	bgcolor={},		% Quitar color de fondo del código inline
	breaklines,		% Permitir cortar la linea en un linebreak
	breakafter=_,		% Permitir cortar palabras en las barras bajas
}

\hypersetup{
	colorlinks=true,	% Links del índice visibles
	linktoc=all,		% Links del índice activados
	linkcolor=black,	% Links del índice negros
}

%----------------------------------------------------------------------------------------
%	PORTADA
%----------------------------------------------------------------------------------------

\begin{document} 

\begin{titlepage} % Elimina los encabezados y los piés de página de la portada

	\centering
	
	\scshape % Cambia las minúsculas por mayúsculas pequeñas
	
	\vspace*{\baselineskip} % Espacio en blanco en la parte superior de la página
	
	%------------------------------------------------
	%	Título
	%------------------------------------------------
	
	\rule{\textwidth}{1.6pt}\vspace*{-\baselineskip}\vspace*{2pt} % Linea gruesa superior
	\rule{\textwidth}{0.4pt} % Linea fina superior
	
	\vspace{0.75\baselineskip} % Espacio en blanco sobre el título
	
	{\LARGE \textbf{PRÁCTICA 2}} % Título
	
	\vspace{0.75\baselineskip} % Espacio en blanco bajo el título
	
	\rule{\textwidth}{0.4pt}\vspace*{-\baselineskip}\vspace{3.2pt} % Linea fina inferior
	\rule{\textwidth}{1.6pt} % Linea gruesa superior
	
	\vspace{2\baselineskip} % Espacio en blanco bajo el bloque de título
	
	%------------------------------------------------
	%	Subtítulo
	%------------------------------------------------
	
	\textbf{Administración y Gestión de Bases de Datos} % Nombre de la asignatura
	
	\vspace*{3\baselineskip} % Espacio en blanco bajo el subtítulo
	
	%------------------------------------------------
	%	Autores
	%------------------------------------------------
	
	\textbf{Escrito por}
	
	\vspace{0.5\baselineskip} % Espacio en blanco sobre los autores
	
	{\scshape\Large \textbf{Alejandro Fernández de la Puebla Ugidos\\ Miguel Hermoso Mantecón \\ Carlos Lafuente Sanz \\}} % Lista de autores
	
	\vspace{1.0\baselineskip} % Espacio en bajo los autores
	
	\textit{\textbf{Universidad Politécnica de Madrid \\}} % Universidad

	\vspace{0.25\baselineskip} % Espacio en blanco entre la universidad y el campus

	\textit{\textbf{ETSISI}} % Campus
	
	\vfill % Espacio en blanco central
	
	%------------------------------------------------
	%	Fecha y publisher
	%------------------------------------------------
	
	\textbf{11 de diciembre de 2022} % Fecha
	
	\vspace{0.5\baselineskip} % Espacio antes del publisher

	\plogo % Logo del publisher

\end{titlepage}

%----------------------------------------------------------------------------------------
%	CONTENIDO
%----------------------------------------------------------------------------------------

\renewcommand*\contentsname{Índice} % Título del índice en español

\setcounter{tocdepth}{3} % Mostrar hasta subsections en el índice

\tableofcontents % Índice

\newpage

%------------------------------------------------
%	Parte 1
%------------------------------------------------
	
\section{Parte 1}

\subsection{Crear los roles y los usuarios}

Se quieren crear los roles \mintinline{mysql}{gestor}, \mintinline{mysql}{compradorJuegos} y \mintinline{mysql}{dependiente} con las siguientes especificaciones:

El rol \mintinline{mysql}{gestor} podrá realizar cualquier operación LMD sobre las tablas y tendrá permiso de propagación de privilegios. Además, los usuarios con este rol podrán operar (crear, modificar, etc.) con los roles, usuarios y privilegios que se abordan en los siguientes puntos.

El rol \mintinline{mysql}{compradorJuegos} podrá visualizar todas las tablas de la BD y también podrá dar de alta nuevos videojuegos que ha adquirido (operando sobre la tabla juegos).

El rol \mintinline{mysql}{dependiente} podrá visualizar todas las tablas de la BD, dar de alta nuevos clientes y modificar clientes existentes (operando sobre la tabla clientes), así como indicar que un cliente ha alquilado un videojuego añadiendo un nuevo registro (operando sobre la tabla clientes\_juegos).

Tambíen se quieren crear los usuarios \mintinline{mysql}{gohan} con rol \mintinline{mysql}{gestor}, \mintinline{mysql}{vegeta} con el rol \mintinline{mysql}{compradorJuegos}, \mintinline{mysql}{videl} con el rol \mintinline{mysql}{compradorJuegos}, \mintinline{mysql}{trunks} con el rol \mintinline{mysql}{dependiente} y \mintinline{mysql}{goku} con el rol \mintinline{mysql}{dependiente}.

Esto se realiza con los siguientes scripts:

\inputminted{mysql}{\scriptdir scripts_roles/gestor.sql}

\inputminted{mysql}{\scriptdir scripts_roles/comprador_juegos_privileges_and_role_creation.sql}

\inputminted{mysql}{\scriptdir scripts_roles/dependiente_privileges_and_role_creation.sql}

\subsection{Comprobar los cambios}

Con el usuario root se quiere comprobar que los cambios anteriores se han realizado correctamente. Eso se hizo de dos formas distintas con el siguiente script:

\inputminted{mysql}{\scriptdir scripts_consulta_privilegios/check_privileges.sql}

El primer método consulta el catálogo. El segundo directamente usa la sentencia de consulta de privilegios de MySQL.

Como tercer método se puede usar la interfaz de MySQLWorkbench:

\includegraphics[width=\textwidth]{\imagedir privilegios_01.png}
\includegraphics[width=\textwidth]{\imagedir privilegios_02.png}

\subsection{Probar las sentencias}

Cada uno de los usuarios ejecutó unas sentencias para probar sus privilegios. Estos son los scripts que se ocupan de ello y cada uno especifica que usuario debe ejecutarlo:

\inputminted{mysql}{\scriptdir scripts_consulta_privilegios/gohan_queries.sql}
\inputminted{mysql}{\scriptdir scripts_consulta_privilegios/vegeta_queries.sql}
\inputminted{mysql}{\scriptdir scripts_consulta_privilegios/videl_queries.sql}
\inputminted{mysql}{\scriptdir scripts_consulta_privilegios/trunks_queries.sql}
\inputminted{mysql}{\scriptdir scripts_consulta_privilegios/goku_queries.sql}

\subsection{Revocar los permisos de Goku}

Debido a un cambio organizativo, es necesario quitarle los permisos de inesrción y modificación a Goku. Para realizarlo se utiliza el siguiente script:

\inputminted{mysql}{\scriptdir scripts_roles/revoke_goku.sql}

Se le quita el rol de dependiente y se le ponen privilegios personalizados para cumplir los requisitos.

%------------------------------------------------
%	Parte 2
%------------------------------------------------

\section{Parte 2}

\subsection{Crear las vistas para freezer}

Ahora se crea al usuario freezer y las vistas que podrá usar:

\inputminted{mysql}{\scriptdir scripts_roles/freezer_and_views_creation.sql}

\subsection{Crear consultas de freezer}

A continuación se crean las consultas especificadas en el enunciado para freezer sobre las vistas:

\inputminted{mysql}{\scriptdir scripts_consulta_privilegios/freezer_queries.sql}

Se añade \mintinline{mysql}{SHOW PROFILES} para ver los tiempos de ejecución de las consultas.

Para que lo haga root sobre las tablas directamente se crea también el siguiente script:

\inputminted{mysql}{\scriptdir scripts_consulta_privilegios/root_queries.sql}

\subsection{Ejecutar las consultas con root}

Los tiempos de ejecutar las consultas con el usuario root sobre las tablas son los siguientes:

\begin{itemize}
	\item Ejecución de A1: T = 0.2714 segundos.
	\item Ejecución de A2: T = 0.0427 segundos.
	\item Ejecución de A3: T = 0.0188 segundos.
	\item Ejecución de A4: T = 0.0156 segundos.
	\item Ejecución de B1: T = 0.0299 segundos.
	\item Ejecución de B2: T = 0.0041 segundos.
	\item Ejecución de B3: T = 0.0149 segundos.
\end{itemize}

\subsection{Ejecutar las consultas con freezer}

Los tiempos de ejecutar las consultas con el usuario freezer sobre las vistas son los siguientes:

\begin{itemize}
	\item Ejecución de A1 sobre A: T = 0.2769 segundos.
	\item Ejecución de A2 sobre A: T = 0.0299 segundos.
	\item Ejecución de A3 sobre A: T = 0.0223 segundos.
	\item Ejecución de B1 sobre B: T = 0.0027 segundos.
	\item Ejecución de B2 sobre B: T = 0.0016 segundos.
\end{itemize}

Las consultas A4 y B3 fallan si las vistas no tienen todos los campos que no pueden ser nulos de la tabla original. Para resolver este problema se añaden los campos nulos a las vistas (DNI al conjunto A y Consola al conjunto B). Los tiempos tras este cambio de dichas consultas son:

\begin{itemize}
	\item Ejecución de A4 sobre A: T = 0.0084 segundos.
	\item Ejecución de B3 sobre B: T = 0.0153 segundos.
\end{itemize}

\subsection{Comprobar las consultas de freezer sobre las tablas}

Se puede comprobar que freezer no puede operar directamente sobre las tablas intentando hacer consultas desde su usuario a ellas. Fallan tanto \mintinline{mysql}{SELECT * FROM clientes} como \mintinline{mysql}{SELECT * FROM clientes_juegos}.

\subsection{Comparativa de los métodos}

En cuanto a la eficiencia, la consulta de datos a través de vistas puede ser más eficiente que una sobre la tabla subyacente. Esto es porque las vistas precomputan los resultados de las consultas y los almacenan en una tabla virtual, de forma que permite un acceso más rápido a los datos.

La seguridad también puede verse favorecida por el uso de las vistas. Se pueden crear vistas con privilegios propios y los datos justos para manejar el acceso de cada usuario a ellos de forma mucho más precisa.

%------------------------------------------------
%	Parte 3
%------------------------------------------------

\section{Parte 3}

\subsection{Crear la copia de seguridad}

La copia de seguridad se puede hacer a través de la consola de MySQL o de la interfaz gráfica de MySQLWorkbench.

Para realizarlo a través del terminal se haría lo siguiente:

\mint{bash}{mysqldump -u root PracABD1 > backup_10_12_2022.sql}

Como en nuestro caso la MySQL está en un contendor Docker, primero hace falta acceder a su terminal con:

\mint{bash}{docker exec -it mysql bash}

Para hacerlo a través de la interfaz de MySQL Workbench:

\begin{enumerate}
	\item Se selecciona "Server > Data Export".
	\item Se activa la opción "Export to Self-Contained File".
	\item Se indica el fichero al que se quiere exportar.
	\item Se selecciona el schema deseado.
	\item Se pincha en el boton "Start Export".
\end{enumerate}

\subsection{Cargar la copia de seguridad}

Para cargar la copia de seguridad, simplemente se crea el schema nuevo con \mintinline{mysql}{CREATE SCHEMA recovery_10_12_2022} y se carga a través del terminal con \mintinline{bash}{mysql -u root -p recovery_10_12_2022 < backup 10_12_2022.sql} o por el mismo método que la exportación con MySQLWorkbench pero dándole al botón de importar.

\end{document}
